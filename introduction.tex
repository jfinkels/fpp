\section{Introduction}

% % Foreword %
%
% %% Context (anyone - why now?) %%
%
% What is the current situation, and why is the need so important?
%
According to Downey and Fellows in the Introduction to \autocite{df13}, ``The future of algorithms is multivariate.''
Their suggestion is to view a computational problem as a multivariate object instead of the classical univariate viewpoint.
Understanding the parameterization for a computational problem allows us to break algorithms for the problem into parts and more easily identify the complexity of these parts.

%
% %% Need (readers - why you?) %%
%
% Why is this relevant to the reader, and why does something need to be done?
% (Also reference relevant existing work.)
%
In the world of parallel versus sequential computation, the right parameterization of problem that would classically be considered inherently sequential can yield a highly parallel algorithm.
Little work has been done to provide the framework for proving parameterized parallelizability for classical inherently sequential computational problems; previous work has mostly focused on parameterized tractability for classically intractable computational problems.
Practicioners should be able to take advantage of parallelism where it exists, and this line of research may reveal a hidden capacity for parallelism that was previously unknown.

%
% %% Task (author - why me?) %%
%
% What was undertaken to address the need?
%
With this need in mind, we undertake the first comprehensive examination of the definitions, supporting lemmas, and basic structural theorems about highly parallel parameterized problems and inherently sequential parameterized problems;
%
% %% Object (document - why this document?) %%
%
% What does this document cover?
%
this paper provides such a framework.

%
% % Summary %
%
% %% Findings (author - what?)
%
% What did the work reveal when performing the task?
Our work builds on the parameterized complexity theory framework that was originally intended to study the difference between tractable and intractable parameterized problems.
Among other things, we prove
\begin{itemize}
\item the existence of inherently sequential problems whose parameterized versions are parallelizable (\autoref{sec:parancexample}),
\item the existence of inherently sequential problems whose parameterized versions are not parallelizable under a reasonable assumption (\autoref{sec:parapcompleteness}),
\item the existence of a hierarchy of complete problems interpolating between the parameterized versions of the formula satisfiability problem and the circuit satisfiability problem (\autoref{sec:parawnccompleteness}),
\item an equivalence between parameterized parallel verifiability and classical limited nondeterminism (\autoref{sec:ncwnc}).
\end{itemize}
%
%
% %% Conclusion (readers - so what?)
%
% What did the findings mean for the audience?
%
Altogether, these results demonstrates a strong relationship among the resources time, nondeterminism, and parallelism.
%
% %% Perspective (anyone - what now?)
%
% What should be done next?
We hope this framework inspires researchers to look more closely for paralellizable problems by considering parameterizations of classical problems previously considered inherently sequential.
