\section{Fixed-parameter tractability}

% % Foreword %
%
% %% Context (anyone - why now?) %%
%
% What is the current situation, and why is the need so important?
%
In classical complexity, there are limits to parallel computation; for some computational problems, adding more processors does not help solve the problem significantly more quickly.
%
% %% Need (readers - why you?) %%
%
% Why is this relevant to the reader, and why does something need to be done?
% (Also reference relevant existing work.)
%
With the definition of parameterized parallel computation given in the previous section, we now must ask whether there are similar limitations for parameterized problems.
%
% %% Task (author - why me?) %%
%
% What was undertaken to address the need?
%
We adapt our understanding of classical complexity to answer this question in the affirmative.
%
% %% Object (document - why this document?) %%
%
% What does this document cover?
%
%
This section demonstrates that under an appropriate parameterization, the natural inherently sequential classical problems become inherently sequential parameterized problems.

% % Summary %
%
% %% Findings (author - what?)
%
% What did the work reveal when performing the task?
%
The definition of $\para \P$ (also known as $\FPT$) is analogous to that of $\para \NC$.
Unlike for $\para \NC$, however, there are numerous examples of natural parameterized problems in $\para \P$; see the listing in \autocite{cesati06}, for example.

\begin{definition}
  A parameterized problem $(Q, \kappa)$ is in $\para \P$ if there is a deterministic Turing machine $M$, a polynomial $p$, and a computable function $f$ such that $M$ decides $Q$ within $f(\kappa(x)) p(n)$ steps.
\end{definition}

We prove the existence of $\para \P$-complete problems, which are inherently sequential parameterized problems.
These problems are parameterized versions of the well-known bounded halting problem for deterministic Turing machines and the Boolean circuit evaluation problem.
This means that there are parameterized problems for which adding more processors provides no significant speedup in (parallel) time required to find a solution.
%
% %% Conclusion (readers - so what?)
%
% What did the findings mean for the audience?
%
Contrast this parameterized inherent sequentiality with the highly parallelizable parameterized versions of $\P$-complete problems in the previous section.
%
% %% Perspective (anyone - what now?)
%
% What should be done next?
In light of this, we recommend that researchers consider $\para \P$-completeness when determining membership of a parameterized problem in $\para \P$ (also known as $\FPT$).

%% Consider the parameterized high degree subgraph problem.

%% \begin{definition}[$p \dash \textsc{High Degree Subgraph}$, aka $\pHDS$]
%%   \mbox{} \\
%%   \begin{tabular}{r p{9.2cm}}
%%     \textbf{Instance:} & undirected graph $G$, positive integer $d$. \\
%%     \textbf{Parameter:} & $d$. \\
%%     \textbf{Question:} & Does $G$ have a vertex-induced subgraph of minimum degree at least $d$?
%%   \end{tabular}
%% \end{definition}

%% The underlying decision problem is in $\P$, so the parameterized problem is fixed-parameter tractable.

%% \begin{theorem}[{\autocite{am84}}]
%%   $\pHDS$ is in $\para \P$.
%% \end{theorem}

%% The first two slices of the problem are in $\NC$, whereas all other slices are $\P$-complete.

%% \begin{theorem}[{\autocite{am84}}]
%%   \mbox{}
%%   \begin{enumerate}
%%   \item Both $pHDS_1$ and $\pHDS_2$ are in $\NC$.
%%   \item For each positive integer $d$ greater than two, $\pHDS_d$ is $\P$-complete.
%%   \end{enumerate}
%% \end{theorem}

%% %% As a maximization problem, the problem is approximable to within a certain constant factor, but inapproximable beyond that factor (unless $\NC = \P$) \autocite{am84}.

%% \todo{The proof that the high degree subgraph problem is $\P$-complete does not translate to proving that the parameterized high degree subgraph problem is $\para \P$-complete; for that reduction to work, the parameterization would need to be on the size of the graph, not the degree $d$.}

%% \todo{A reduction to place this problem in $\para \EP$ does not seem possible, since it seems hard to encode the degree of a subgraph in that way?}

\subsection{Completeness in \texorpdfstring{$\para \P$}{paraP}}
\label{sec:parapcompleteness}

% % Foreword %
%
% %% Context (anyone - why now?) %%
%
% What is the current situation, and why is the need so important?
%
We define $\P$-completeness so that problems that are $\P$-complete are unlikely to see a significant decrease in time complexity when parallelism is allowed, under the assumption that $\NC \neq \P$.
Let us define $\para \P$-completeness similarly, so that $\para \P$-complete problems are unlikely to see a significant decrease in ``parameterized'' time complexity when ``parameterized'' parallelism is allowed, under the assumption that $\para \NC \neq \para \P$.
%
% %% Need (readers - why you?) %%
%
% Why is this relevant to the reader, and why does something need to be done?
% (Also reference relevant existing work.)
%
We already know that each $\P$-complete problem induces a $\para \P$-complete problem with a trivial parameterization \autocite[Proposition~14]{fg03}, however we are interested in natural problems with non-trivial parameterizations.
%
% %% Task (author - why me?) %%
%
% What was undertaken to address the need?
%
We prove the existence of nontrivial inherently sequential parameterized problems that are not in $\para \NC$ (under the assumption $\para \NC \neq \para \P$) and whose underlying decision problems are $\P$-complete, complementing \autoref{sec:pcompletefpp}, which proves the existence of parameterized problems in $\para \NC$ whose underlying decision problems are $\P$-complete.
%
% %% Object (document - why this document?) %%
%
% What does this document cover?
%
%

% % Summary %
%
% %% Findings (author - what?)
%
% What did the work reveal when performing the task?
%
Specifically, we show that parameterized versions of the circuit evaluation problem and the bounded halting problem are $\para \P$-complete
% %% Conclusion (readers - so what?)
%
% What did the findings mean for the audience?
%
This means that the classical $\P$-complete problems do in fact become $\para \P$-complete, but it is important to choose the correct parameterization (which is not obvious in some cases).
Furthermore, this complements the work of \autocite{est15}, in which the authors show complete problems for several other parameterized complexity classes.
%
% %% Perspective (anyone - what now?)
%
% What should be done next?
These results provide fundamental limits to parallelizability for parameterized problems.

We can really only say that our $\para \P$-complete problems are inherently sequential under the assumption that $\para \NC \neq \para \P$.
This assumption is reasonable because it is equivalent to the inequality $\NC \neq \P$.

\begin{proposition}\label{prop:ncp}
  $\para \NC = \para \P$ if and only if $\NC = \P$.
\end{proposition}
\begin{proof}
  The proof is trivial if we use the definitions of the complexity class $\para \mathcal{C}$ as the class of all parameterized problems $(Q, \kappa)$ for which there is a language $L$ in the complexity class $\mathcal{C}$ such that $x \in Q$ if and only if $(x, 1^{f(\kappa(x))}) \in L$.
  See \autocite[Proposition~8]{fg03}, for example.
\end{proof}

\begin{definition}[$\para \P$-completeness]
  A parameterized problem $(Q, \kappa)$ is \emph{$\para \P$-hard} if for each parameterized problem $(R, \lambda)$, there is a $\para \NC$ many-one reduction from $(R, \lambda)$ to $(Q, \kappa)$.
  If furthermore $(Q, \kappa)$ is in $\para \P$, then it is \emph{$\para \P$-complete}.
\end{definition}

\begin{proposition}
  If a $\para \P$-complete problem is in $\para \NC$, then $\para \NC = \para \P$.
\end{proposition}
\begin{proof}
  Follows from the downward closure of $\para \NC$ under $\para \NC$ many-one reductions.
\end{proof}

The following problem is adapted from \textsc{Short Deterministic Turing Machine Computation} in \autocite{cesati06}.

\begin{definition}[$p \dash \textsc{Bounded Halting Problem}$, aka $\pBHP$]
  \mbox{} \\
  \begin{tabular}{r p{9.2cm}}
    \textbf{Instance:} & deterministic Turing machine $M$, binary string $x$ of length $n$, positive integer $t$ in unary, positive integer $c$. \\
    \textbf{Parameter:} & $t / n^c$ \\
    \textbf{Question:} & Does $M$ accept $x$ within $t$ steps?
  \end{tabular}
\end{definition}

\begin{theorem}
  $\pBHP$ is $\para \P$-complete.
\end{theorem}
\begin{proof}
  The underlying decision problem is in $\P$ (by a standard simulation on the deterministic universal Turing machine), so the parameterized problem is in $\para \P$.
  To show $\para \P$-hardness, we use a generic reduction.
  Let $(Q, \kappa)$ be an arbitrary parameterized problem in $\para \P$ and let $M$ be the deterministic Turing machine that decides $Q$ in time $f_M(k) n^c$ for some (circuit-)computable function $f_M$ and some positive integer $c$.
  The reduction is $x \mapsto (M, x, 1^{f_M(k) n^c}, c)$.
  This is computable by a (nonuniform) circuit family of constant depth and size $f(k) n^{O(1)}$, where $f$ is a circuit-computable function.
  The parameter of the reduced instance is $f_M(k) n^c / n^c$, or simply $f_M(k)$, which satisfies the parameter bound required by the definition of $\para \NC$ many-one reduction.
  Therefore we conclude that $\pBHP$ is $\para \P$-complete.
\end{proof}

The following problem is a modification of \textsc{BS-BD-CVP} from \autocite{cd98}.

\begin{definition}[$p\dash\textsc{Small Circuit Evaluation}$, aka $\pSCE$]
  \mbox{} \\
  \begin{tabular}{r p{9.2cm}}
    \textbf{Instance:} & Boolean circuit $C$ on $n$ inputs, binary string $x$ of length $n$, positive integer $k$, positive integer $\alpha$, multi-output Boolean circuit $f$ with size and depth of $C$ at most $f(k) n^\alpha$. \\
    \textbf{Parameter:} & $k$ \\
    \textbf{Question:} & Does $C(x) = 1$?
  \end{tabular}
\end{definition}

This theorem is related to \autocite[Corollary~2]{cd98}, where the authors prove that the \textsc{BS-BD-CVP} problem is complete for the class $\PNC$ (a class that exists between $\para \NC$ and $\para \P$) under $\para \NC$ many-one reductions.
While their reduction is a generic reduction, ours is a reduction from the parameterized bounded halting problem.

\begin{theorem}\label{thm:psce}
  $\pSCE$ is $\para \P$-complete.
\end{theorem}
\begin{proof}
  Membership in $\para \P$ is straightforward to prove: use the natural algorithm for evaluating a circuit which can be performed in linear time with respect to the size of the circuit.
  We must also compute $f(k) n^\alpha$ and compare it with the size and depth of the circuit $C$.
  Both of these are polynomial-time algorithms with respect to the size of the input, and hence the problem is in $\para \P$.

%% %%%% Here we showed a generic reduction.
%%   Now we prove $\FPT$-hardness.
%%   Let $(Q, \kappa)$ be an arbitrary parameterized problem in $\FPT$, so there is a deterministic Turing machine $M$ that decides $Q$ in time $f(k) n^\alpha$ for some (circuit-)computable function $f$ and positive integer $\alpha$.
%%   Let $C_{M, n}$ be the Boolean circuit that correctly simulates the action of $M$ on inputs of size $n$ with size $O(f(k) n^\alpha)^2$ and depth $O(f(k) n^\alpha)$; such a circuit exists due to \autocite{???}.
%%   Define the circuit family $\{R_{n, k}\}$ as follows.
%%   \[
%%   R_{n, k}(x) = (C_{M, n}, x, 2\alpha, \alpha, f^2, f, k),
%%   \]
%%   where the natural number $n = |x|$, the positive integer $k = \kappa(x)$, and the function $f^2$ denotes the function $x \mapsto (f(x))^2$.
%%   Let us confirm that the reduced instance is well-formed.
%%   \begin{itemize}
%%   \item $C_{M, n}$ is a Boolean circuit on $n$ inputs.
%%   \item The size of $C_{M, n}$ is $f^2(k) n^{2 \alpha}$.
%%   \item The depth of $C_{M, n}$ is $f(k) n^\alpha$.
%%   \item Both $f^2$ and $f$ are circuit-computable since $f$ is circuit-computable.
%%   \end{itemize}
%%   (There are some constants in the size and depth bounds that we have ignored, but those can be incorporated into the circuit-computable function $f$.)
%%   The reduction is correct because
%%   \[
%%   x \in Q \iff M \text{ accepts } x \iff C_{M, n}(x) = 1.
%%   \]

%%   Next, we show that $R_{n, k}$ has the appropriate size and depth bounds to be a valid $\FPP$ many-one reduction.
%%   The only computation that needs to be done by the circuit is to compute $2 \alpha$ from $\alpha$ and to compute the circuit for $f^2$ from the circuit for $f$.
%%   However, since both $\alpha$ and $f$ are entirely independent of $n$, the size and depth are constants with respect to $|x|$.
%%   The remaining elements of the tuple (namely $C_{M, n}$, $\alpha$, $f$, and $k$) are hardcoded directly in the circuit $R_{n, k}$ and are output directly with no further computation.
%%   The input $x$ is output directly as well.
%%   The overall size of the circuit is therefore of the form $s(k) n^{O(1)}$, and the depth of the form $d(k) n^{O(1)}$.

%%   Finally, for the parameter bound, the parameter of $R_{n, k}(x)$ is $k$, which is exactly $\kappa(x)$.
%%   Therefore we have shown that the circuit family $\{R_{n, k}\}$ is a correct $\FPP$ many-one reduction from an arbitrary parameterized problem $(Q, \kappa)$ to $\pSCE$.
%% \end{proof}

%% \begin{theorem}
%%   $\pBHP \equiv_m^{\FPP} p \dash \textsc{Short Circuit Evaluation}$.
%% \end{theorem}
%% \begin{proof}
  Now we prove $\para \P$-hardness.
  The reduction from $\pBHP$ is
  \[
  (M, x, 1^t, c) \mapsto (C_M, x, k, \alpha, f),
  \]
  where
  \begin{itemize}
  \item $C_M$ is the standard circuit of size $O(t^2)$ and depth $O(t)$ simulating $t$ steps of the action of $M$ on inputs of length $n$,
  \item $x$ is copied directly from the input,
  \item $k = t / n^c$.
  \item $\alpha = 2c$,
  \item $f$ is the function $x \mapsto x^2$,
  \end{itemize}
  This reduction is computable in the appropriate size and depth bounds, and its correctness follows from the correctness of the standard deterministic Turing machine-to-circuit reduction.
  To check that the reduced instance is well-formed, let us verify that the circuit $C_M$ meets the size and depth requirements.
  The size of $C_M$ is $O(t^2)$, which is $O(((t / n^c) n^c)^2)$, or simply $f(k) n^\alpha$.
  Similarly the depth of $C_M$ is $O(t)$, which is smaller than $O(t^2)$, and thus bounded above by $f(k) n^\alpha$ as well.
  (There are some constants in the size and depth bounds that we have ignored, but those can be incorporated into the definition of $f$.)
  Finally, the parameter in the original instance, $t / n^c$, is exactly the parameter of the reduced instance, so this reduction meets the necessary paramater bound.
  Therefore we have shown a correct $\para \NC$ many-one reduction from a $\para \P$-complete problem.
\end{proof}

As expected, if any of these $\para \P$-complete problems are fixed-parameter parallelizable, then $\para \NC = \para \P$.

It seems that most $\P$-complete problems will end up being $\para \P$-complete under this notion of completeness.
This doesn't really help us distinguish between different $\P$-complete problems based on how fixed-parameter parallelizable they are.
In the next subsection, we try a different approach.

\subsection{Parameterized complexity of efficient verification}

% % Foreword %
%
% %% Context (anyone - why now?) %%
%
% What is the current situation, and why is the need so important?
%
There is a nice relationship between the circuit satisfiability (decision) problem and the circuit evaluation problem: an algorithm for the latter is a verification procedure for the former, given a satisfying assignment.
The relationship between these two problems in particular reflect the general relationship between $\NP$ and $\P$.
%
% %% Need (readers - why you?) %%
%
% Why is this relevant to the reader, and why does something need to be done?
% (Also reference relevant existing work.)
%
Does the same sort of relationship hold for the parameterized versions of these classes, $\para \WP$ and $\para \P$?
%
% %% Task (author - why me?) %%
%
% What was undertaken to address the need?
%
We define a new parameterized complexity class to address this question.
%
% %% Object (document - why this document?) %%
%
% What does this document cover?
%
%

% % Summary %
%
% %% Findings (author - what?)
%
% What did the work reveal when performing the task?
%
We show that our parameterized complexity class, $\para \EP$, the verification class of $\para \WP$, is a subclass of $\para \P$.
%
% %% Conclusion (readers - so what?)
%
% What did the findings mean for the audience?
%
We are unable to show that is equal to $\para \P$, so our intuition does not yet match our definitions.
%
% %% Perspective (anyone - what now?)
%
% What should be done next?
What remains is to show a natural problem in this class and to determine whether it equals $\para \P$ or not.

We start by considering the parameterized weighted circuit satisfiability and circuit evaluation problems.
A circuit is \emph{$k$-satisfiable} if there is a satisfying assignment of Hamming weight exactly $k$.

\begin{definition}[$p \dash \textsc{Circuit } k \dash \textsc{Satisfiability}$, aka $\pCkSAT$]
  \mbox{} \\
  \begin{tabular}{r p{9.2cm}}
    \textbf{Instance:} & Boolean circuit $C$, natural number $k$. \\
    \textbf{Parameter:} & $k$. \\
    \textbf{Question:} & Is $C$ $k$-satisfiable?
  \end{tabular}
\end{definition}

The corresponding parameterized weighted circuit evaluation problem would then be as follows.
Let $\|x\|_1$ denote the Hamming weight (that is, the number of ones) in $x$.

\begin{definition}[$p \dash \textsc{Circuit } k \dash \textsc{Evaluation}$, aka $\pCkE$]
  \mbox{} \\
  \begin{tabular}{r p{9.2cm}}
    \textbf{Instance:} & Boolean circuit $C$, binary string $x$, natural number $k$. \\
    \textbf{Parameter:} & $k$. \\
    \textbf{Question:} & Does $\|x\|_1 = k$ and $C(x) = 1$?
  \end{tabular}
\end{definition}

This is a little silly, since the Hamming weight of $x$ can be computed easily (in $\NC^1$ but not in $\AC^0$), but it is technically the ``verification'' problem corresponding to the satisfiability problem above.
Instead, we use a problem that is a little less silly while still equivalent under $\para \NC^1$ many-one reductions.
%% \todo{Reference for circuit computing Hamming weight; see \url{http://ieeexplore.ieee.org/xpls/abs_all.jsp?arnumber=4781532}}

\begin{definition}[$p \dash \textsc{Weighted Circuit Evaluation}$, aka $\pWCE$]
  \mbox{} \\
  \begin{tabular}{r p{9.2cm}}
    \textbf{Instance:} & Boolean circuit $C$, binary string $x$. \\
    \textbf{Parameter:} & $\|x\|_1$. \\
    \textbf{Question:} & Does $C(x) = 1$?
  \end{tabular}
\end{definition}

%% \begin{proposition}
%%   $\pCkE \equiv_m^{\para \NC^1} \pWCE$.
%% \end{proposition}
%% \begin{proof}
%%   Let $\kappa_1$ denote the parameterization of the problem on the left and $\kappa_2$ the parameterization of the problem on the right.
%%   The right-to-left reduction is simply $(C, x) \mapsto (C, x, \|x\|_1)$.
%%   The correctness as well as the size, depth, and parameterization bounds are straightforward to prove.

%%   The circuit family computing the left-to-right reduction is $\{R_{m, k}\}$, defined by
%%   \[
%%   R_{m, k}(C, x, k) =
%%   \begin{cases}
%%     (C, x) & \text{if } \|x\|_1 = k \\
%%     (\textsc{not}, 1) & \text{otherwise},
%%   \end{cases}
%%   \]
%%   where $m$ denotes the overall size of the input and \textsc{not} is the circuit which is a single negation gate.

%%   In order to prove correctness, first suppose $C(x) = 1$ and $\|x\|_1 = k$.
%%   By construction, $R(C, x, k) = (C, x)$, and $C(x) = 1$, as required.
%%   For the converse, there are two cases to consider.
%%   First, if $C(x) \neq 0$, then the output of $R(C, x, k)$ will be a circuit that outputs zero regardless of the value of $\|x\|_1$.
%%   Second, if $\|x\|_1 \neq k$, then the output of $R(C, x, k)$ will be $(\textsc{not}, 1)$, which is a circuit that outputs zero.
%%   In either case, $R(C, x, k)$ is not in $\pwce$, as required.
%%   Thus the reduction is correct.

%%   The size and depth bounds are trivial since the circuit $R_{m, k}$ essentially just copies its input to its output.
%%   Finally, for the bound on the parameterization in the reduced instance, there are two cases.
%%   If $\|x\|_1 = k$, then
%%   \[
%%   $\kappa_2(R(C, x, k)) = \kappa_2(C, x) = \|x\|_1 = k = \kappa_1(C, x).
%%   \]
%%   Otherwise,
%%   \[
%%   \kappa_2(R(C, x, k)) = \kappa_2(\textsc{not}, 1) = 1 \leq \kappa_1(C, x).
%%   \]
%%   In either case, the parameterization of the reduced instance is bounded above by the parameterization of the original instance.
%%   Therefore we have shown the left-to-right $\FPP$ reduction.
%% \end{proof}

In the setting of decision problems, we know that $\NP$ can be characterized as the closure of the circuit satisfiability problem under polynomial-time many-one reductions,
\[
\NP = \cl{\textsc{Circuit Satisfiability}}^{\leq_m^\P},
\]
and $\P$ as the closure of the circuit evaluation problem under $\NC^1$ many-one reductions,
\[
\P = \cl{\textsc{Circuit Evaluation}}^{\leq_m^{\NC^1}}.
\]
In the setting of parameterized problems, the class $\para \WP$ can be characterized as the closure of the parameterized weighted circuit satisfiability problem under fixed-parameter tractable many-one reductions,
\[
\para \WP = \cl{\pCkSAT}^{\leq_m^{\para \P}}.
\]
Following the above pattern, we define a new class as the closure of the parameterized weighted circuit evaluation problem under fixed-parameter parallelizable many-one reductions,
\[
\para \EP = \cl{\pWCE}^{\leq_m^{\para \NC}}
\]
(``E'' for evaluation).
%% \todo{This is a name conflict with an existing class called $\EP$.}

Since the underlying decision problem, the problem of evaluating a circuit on a given input, is in $\P$, the parameterized problem $\pWCE$ is trivially in $\para \P$.
Since $\para \NC$ reductions compose, $\para \NC$ is a subset of $\para \P$, and $\para \P$ is closed under $\para \P$ reductions, we conclude that $\para \EP$ is a subset of $\para \P$.

\begin{theorem}
  $\para \EP \subseteq \para \P$.
\end{theorem}

Is $\para \EP = \para \P$?
The standard simulation of a deterministic Turing machine by a circuit, as in \autocite{ladner75}, for example, fails to provide a $\para \NC$ many-one reduction to the parameterized weighted circuit evaluation problem, since the natural reduction would be of the form $x \mapsto (C, x, \|x\|_1)$, but the parameter value $\|x\|_1$ is not necessarily bounded by a function of $\kappa(x)$.
The same issue prevents us from showing that $\para \NC \subseteq \para \EP$.

%% \todo{Show something bad happens if $\para \EP = \para \P$.}

%% \todo{Show a problem that is known to be in $\para \P$ (for example one of the deterministic Turing machine computation problems maybe) is actually in $\para \EP$.}
