\section{Fixed-parameter parallelizability}
\label{sec:pcompletefpp}

The $\P$-complete problems are considered tractable but inherently sequential; adding more processors does not provide any significant reduction in the time required to find a solution, even thought they can be solved in polynomial time.
But can the lack of parallelizability for these problems depend on a parameterization of the problem, thereby isolating some of the ``sequential-ness'' of the problem in the value of the parameter?

\subsection{Definition of \texorpdfstring{$\para \NC$}{paraNC}}

The $\para$ ``operator'' defined in \autocite{fg03} applies generically to an arbitrary complexity class as follows.
If $\mathcal{C}$ is a class of decision problems, then $\para \mathcal{C}$ is the class of parameterized problems $(Q, \kappa)$ for which there is a decision problem $L \in \mathcal{C}$ and a computable function $f$ such that $x$ in $Q$ if and only if $(x, 1^{f(\kappa(x))}) \in L$.
When $\mathcal{C} = \NC$ in particular, we get the following equivalent definition.

\begin{definition}[$\para \NC^d$]
  Let $d$ be a natural number.
  A parameterized problem $(Q, \kappa)$ is in the class $\para \NC^d$ if there is a circuit-computable function $f$ and a nonuniform family $\{C_{n, k}\}$ of bounded fan-in Boolean circuits such that for each string $x$,
  \begin{itemize}
  \item $x \in Q$ if and only if $C_{n, k}(x) = 1$, where $n = |x|$ and $k = \kappa(x)$,
  \item $\size(C_{n, k}) \leq f(k) n^{O(1)}$,
  \item $\depth(C_{n, k}) \leq f(k) + O(\log^d n)$.
  \end{itemize}
\end{definition}

If the depth of the circuit is instead bounded by $f(k) O(\log^d n)$, the class is denoted $\para \NC^{d \uparrow}$, a superclass of $\para \NC^d$.
If the circuits are of unbounded fan-in, the classes are $\para \AC^d$ and $\para \AC^{d \uparrow}$, respectively.
The classes $\para \AC^{d \uparrow}$ were first defined in \autocite{bst15}.

\subsection{Example problem in \texorpdfstring{$\para \NC$}{paraNC}}

The parameterized vertex cover problem is in $\para \AC^0$ \autocite{bst15}, but the underlying decision problem is $\NP$-complete.
We are interested in finding a problem in $\para \NC$ whose underlying decision problem is $\P$-complete.
One way to do this is to choose $Q$ to be a $\P$-complete problem and $\kappa$ to be the ``degenerate'' parameterization function $\kappa(x) = |x|$.

\begin{definition}[$p \dash s \dash \textsc{Circuit Evaluation}$]
  \mbox{} \\
  \begin{tabular}{r p{9.2cm}}
    \textbf{Instance:} & Boolean circuit $C$ on $n$ inputs with size $m$ and depth $d$, binary string $x$ of length $n$. \\
    \textbf{Parameter:} & $m$. \\
    \textbf{Question:} & Does $C(x) = 1$?
  \end{tabular}
\end{definition}

\begin{theorem}
  $p \dash s \dash \textsc{Circuit Evaluation}$ is in $\para \NC$ and the underlying decision problem is $\P$-complete.
\end{theorem}
\begin{proof}
  The circuit evaluation problem is $\P$-complete by \autocite{ladner75}.
  Since the parameterization is monotonically increasing with the size of the input, the problem is in $\para \NC$ by the technique of \autocite[Proposition~1.7]{fg06}.
\end{proof}

To find a non-degenerate example, we can parameterize the circuit evaluation problem by depth instead of size.

\begin{definition}[$p \dash d \dash \textsc{Circuit Evaluation}$]
  \mbox{} \\
  \begin{tabular}{r p{9.2cm}}
    \textbf{Instance:} & Boolean circuit $C$ on $n$ inputs with size $m$ and depth $d$, binary string $x$ of length $n$. \\
    \textbf{Parameter:} & $d$. \\
    \textbf{Question:} & Does $C(x) = 1$?
  \end{tabular}
\end{definition}

\begin{theorem}\label{thm:cvpdepth}
  $p \dash d \dash \textsc{Circuit Evaluation}$ is in $\para \AC^{0 \uparrow}$ and the underlying decision problem is $\P$-complete.
\end{theorem}
\begin{proof}
  As stated in the proof of the previous theorem, the circuit evaluation problem is $\P$-complete.
  Evaluating the circuit $C$ of size $m$ and depth $d$ on inputs $x$ can be performed by the depth-universal circuit $U$ of \autocite{ch85}.
  The size of $U$ is $O(m)$ and the depth is $d$, so there is a function $f$ such that the size is bounded by $f(d) m^{O(1)}$ and the depth by $f(d)$.
  Therefore the circuit evaluation problem parameterized by circuit depth is in $\para \AC^{0 \uparrow}$.
\end{proof}

For a problem with a standard parameterization derived from an optimization problem (see \autoref{def:standard} below), consider the ``depth of ones'' problem.
The circuit evaluation problem is a special case of the depth of ones problem if we choose $k$ to be the depth of the circuit $C$.

\begin{definition}[$p \dash \textsc{Ones Depth}$]
  \mbox{} \\
  \begin{tabular}{r p{9.2cm}}
    \textbf{Instance:} & Boolean circuit $C$ on $n$ inputs with size $m$ and depth $d$, binary string $x$ of length $n$, positive integer $k$. \\
    \textbf{Parameter:} & $k$. \\
    \textbf{Question:} & When evaluating $C(x)$, do ones propagate to depth at least $k$?
  \end{tabular}
\end{definition}

As an optimization problem, the depth of ones problem is inapproximable up to any constant factor by any $\NC$ circuit, unless $\NC = \P$ \autocite{ks88}.
Contrast this with the complexity of the corresponding parameterized problem.

\begin{theorem}
  $p \dash \textsc{Ones Depth}$ is in $\para \AC^{0 \uparrow}$ and the underlying decision problem is $\P$-complete.
\end{theorem}
\begin{proof}
  Computing the depth of ones in a circuit is $\P$-complete \autocite{ks88} (see also \autocite[Problem~A.1.10]{ghr95}).
  The naïve algorithm for solving this problem is to take the subcircuit consisting of all gates starting from the inputs and extending through layer $k$, evaluating that (multi-output) circuit, then applying a single \textsc{or} gate to decide whether any of the gates at layer $k$ evaluated to one.
  For each gate at layer $k$, use an instance of the depth-universal circuit to evaluate the single-output circuit induced by that gate.
  This yields a circuit of depth $O(k)$ and size $f(k) m^{O(1)}$ for some $f$, where $m$ is the size of the circuit given as input.
  Therefore this problem is in $\para \AC^{0\uparrow}$.
\end{proof}

\subsection{Relationship between \texorpdfstring{$\para \NC$}{paraNC} and \texorpdfstring{$\NC$}{NC}}

The lemmas in this section allow us to transform, in certain circumstances, a highly parallel algorithm for a decision problem into a fixed-parameter parallelizable one for a parameterized problem, or vice versa.
They will be used in later sections to provide evidence against the collapse of larger complexity classes to $\para \NC$.

\begin{lemma}\label{lem:upperinverse}
  Suppose $i$ is an unbounded, nondecreasing, circuit-computable function.
  Then there is an unbounded, nondecreasing, circuit-computable function $f_i$ such that for each natural number $n$, we have $f_i(i(n)) \geq n$.
  (We call $f_i$ the ``upper inverse'' of $i$.)
  \todo{Make this look like the inverse lemma.}
\end{lemma}
\begin{proof}
  \todo{Fill me in}
\end{proof}

\begin{definition}\label{def:spreduction}
  Suppose $d$ is a natural number, $(Q, \kappa)$ is a parameterized problem, and $Q'$ is a decision problem.
  There is a \emph{small parameter $\NC^d$ many-one reduction} from $(Q, \kappa)$ to $Q'$ if there is a nondecreasing, unbounded, circuit-computable function $h$ and an $\NC^d$ family of circuits $\{R_n\}_{n \in \mathbb{N}}$ such that for each string $x$ of length $n$ with $\kappa(x) \leq h(n)$, we have $x \in Q$ if and only if $R_n(x) \in Q'$.
\end{definition}

%% This lemma is analogous to \autoref{thm:eventually}.
This lemma demonstrates that the closure of $\NC$ under small parameter reductions is $\para \NC$.

\begin{lemma}\label{lem:spreduction}
  Suppose $d$ is a natural number, $(Q, \kappa)$ is a parameterized problem, and $Q'$ is a decision problem.
  If $Q$ is circuit-decidable with uniform size and depth, $Q'$ is in $\NC^d$, and there is a small parameter $\NC^d$ many-one reduction from $(Q, \kappa)$ to $Q'$, then $Q$ is in $\para \NC^d$.
\end{lemma}
\begin{proof}
  Let $h$ be the function that defines the upper bound on the parameter, below which there is an $\NC^d$ many-one reduction from $Q$ to $Q'$.
  Let $\{R_n\}$ be the $\NC^d$ circuit family computing the reduction.
  The nonuniform family of circuits $\{A_{n, k}\}$ that decides $(Q, \kappa)$ is defined by
  \[
  A_{n, k} =
  \begin{cases}
    C_n^1 & \text{if } h(n) < k \\
    C_{n'}^2 \circ R_{n} & \text{otherwise},
  \end{cases}
  \]
  where $\{C_n^1\}$ is the family of circuits that decides $Q$ with uniform size and depth, $\{C_n^2\}$ is the family of $\NC^d$ circuits that decides $Q'$, and $n'$ is the number of output bits of $R_n$.
  The correctness of $A_{n, k}$ follows from the correctness of the subsequent circuits.

  If $h(n) \geq k$, then the size and depth of the circuit are polynomial and polylogarithmic in $n$, respectively, because the size and depth of $C_{n'}^2$ and $R_n$ are.
  For the case when $h(n) < k$, consider the upper inverse $i_h$ of $h$ guaranteed by \autoref{lem:upperinverse}.
  By construction, $n \leq i_h(h(n)) < i_h(k)$.
  Now
  \begin{align*}
    \size(A_{n, k}) & = \size(C_n^1) = S(n) \leq S(i_h(k)), \\
    \depth(A_{n, k}) & = \depth(C_n^1) = D(n) \leq D(i_h(k)),
  \end{align*}
  where $S$ and $D$ are the (circuit-computable, nondecreasing) size and depth bounds for the circuit family $\{C_n^1\}$.
  Thus in either case, there is a sufficiently large circuit-computable function $f$ such the size of $A_{n, k}$ is bounded above by $f(k) n^{O(1)}$ and the depth $f(k) + O(\log^d n)$.
\end{proof}

\begin{corollary}\label{cor:sprself}
  Suppose $(Q, \kappa)$ is a parameterized problem, $d$ is a positive integer, and $i$ is an unbounded, nondecreasing, circuit-computable function.
  Let $i(n) \dash Q$ denote the problem of deciding, given $x$ with $\kappa(x) \leq i(|x|)$, whether $x \in Q$.
  If $i(n) \dash Q$ is in $\NC^d$, then $(Q, \kappa)$ is in $\para \NC^d$.
\end{corollary}
\begin{proof}
  The identity function is a small parameter $\NC^d$ many-one reduction from from $(Q, \kappa)$ to $i(n) \dash Q$, thereby proving that $Q$ is in $\para \NC^d$ by the previous lemma.
\end{proof}

This lemma shows that a many-one reduction to a fixed-parameter parallelizable problem can sometimes induce a highly parallel algorithm, if the parameter functions are bounded for the reduced instance.

\begin{lemma}\label{lem:reducetonc}
  Suppose $d$ is a positive integer, $Q$ is a decision problem, and $(Q', \kappa')$ is a parameterized problem.
  Suppose there is an $\NC^d$ many-one reduction from $Q$ to $Q'$, given by the circuit family $\{R_n\}$, and $(Q', \kappa')$ is in $\para \NC^d$ by a circuit family $\{C_{m, k}\}$ of size $f(k) m^{O(1)}$ and depth $f(k) + O(\log^d m)$ on inputs of length $m$.
  If $f(\kappa'(R_n(x))) \leq \min(n^{O(1)}, O(\log^d n))$, then $Q$ is in $\NC^d$.
\end{lemma}
\begin{proof}
  The circuit family that decides $Q$ is $\{A_n\}$, defined by $A_n = C_{m, k} \circ R_n$, where $m$ is the size of the output of $R_n$ and $k = \kappa'(R_n(x))$.
  Since $\size(R_n) = n^{O(1)}$, we have $m = n^{O(1)}$ as well.
  For correctness,
  \[
  x \in Q \iff R_n(x) \in Q' \iff C_{m, k}(R_n(x)) = 1.
  \]
  For size and depth bounds,
  \begin{align*}
    \size(A_n) & = \size(C_{m, k}) + \size(R_n) \\
    & = f(k) m^{O(1)} + n^{O(1)} \\
    & = f(k) n^{O(1)} + n^{O(1)} \\
    & = n^{O(1)} n^{O(1)} + n^{O(1)} \\
    & = n^{O(1)},
  \end{align*}
  and
  \begin{align*}
    \depth(A_n) & = \depth(C_{m, k}) + \depth(R_n) \\
    & = f(k) + O(\log^d m) + O(\log^d m) \\
    & = f(k) + O(\log^d m) \\
    & = f(k) + O(\log^d n) \\
    & = O(\log^d n) + O(\log^d n) \\
    & = O(\log^d n). \qedhere
  \end{align*}
\end{proof}

\subsection{Approximable optimization problems}

Yet another way to do this is to find an optimization problem whose budget problem is $\P$-complete while admitting a highly parallel approximation scheme.

\begin{definition}
  An \emph{optimization problem} $O$ is a four-tuple $(I, S, m, t)$, where $I$ is the set of instances, $S$ is the set of pairs $(x, w)$ where $w$ is a solution for $x$, the function $m$ computes the \emph{measure} (or \emph{objective value}) for such a pair, and $t$ is either $\min$ or $\max$.
\end{definition}

\begin{definition}\label{def:standard}
  The \emph{standard parameterization} of a minimization problem $O$, denoted $p\dash{O}$, is $(Q, \kappa)$, where $Q = \{ (x, k) \, | \, m^*(x) \leq k \}$ and $\kappa(x, k) = k$.
  The inequality is reversed for a maximization problem.
\end{definition}

\begin{definition}
  Suppose $(I, S, m, t)$ is an optimization problem and $(x, y) \in S$.
  The \emph{performance ratio} of the solution $y$ (with respect to $x$), denoted $R(x, y)$, is defined by
  \[
  R(x, y) = \max \left(\frac{m(x, y)}{m^*(x)}, \frac{m^*(x)}{m(x, y)}\right)
  \]
\end{definition}

The performance ratio $R(x, y)$ is a number in the interval $[1, \infty)$.
The closer $R(x, y)$ is to 1, the better the solution $y$ is for $x$, and the closer $R(x, y)$ to $\infty$, the worse the solution.

\begin{definition}
  An \emph{approximation scheme} for an optimization problem is a function $A$ such that for all $x$ and all positive integers $k$ we have $(x, A(x, k)) \in S$ and $R(x, A(x, k)) \leq 1 + \frac{1}{k}$.
\end{definition}

An approximation scheme induces a family of functions, $\{A_k\}_{k \in \mathbb{N}}$, that form progressively better approximations for the optimization problem.

%% \begin{definition}
%%   An optimization problem $O$ has an \emph{$\NC$ approximation scheme} if there is an approximation scheme $A$ for $O$ such that for each $k$, we have $A_k \in \FNC$, where $A_k(x) = A(x, k)$ for each $x$.
%% \end{definition}

\begin{definition}
  Suppose $O$ is an optimization problem with $O = (I, S, m, t)$ with $I$ and $S$ in $\NC$ and $m$ in $\FNC$.
  An optimization problem $O$ is in $\NCAS$ if there is an approximation scheme $A$ for $O$ such that for each $k$, we have $A_k \in \FNC$, where $A_k(x) = A(x, k)$ for each $x$.
  The problem is in $\FNCAS$ if there is an approximation scheme $A$ for $O$ such that $A \in \FNC$ (i.e. on both inputs).
\end{definition}

This definition is adapted from \autocite[Definition~1.31]{fg06}

\begin{definition}
  An optimization problem $O$ is in $\ENCAS$ if there is a circuit family $\{A_{n, k}\}$ and circuit-computable functions $f$ and $g$ such that
  \begin{itemize}
  \item $\{A_{n, k}\}$ is an approximation scheme for $O$,
  \item $\size(A_{n, k}) \leq f(k) n^{O(1)}$,
  \item $\depth(A_{n, k}) \leq g(k) \log^{O(1)} n$.
  \end{itemize}
\end{definition}

\begin{proposition}\label{prop:encas}
  $\FNCAS \subseteq \ENCAS \subseteq \NCAS$.
\end{proposition}

This theorem is an adaptation of \autocite[Theorem~1.32]{fg06}.

\begin{theorem}\label{thm:encasfpp}
  Let $O$ be an optimization problem.
  If $O$ is in $\ENCAS$, then $p\dash{O}$ is in $\para \NC$.
\end{theorem}
\begin{proof}
  Assume without loss of generality that $O$ is a minimization problem; the proof is similar if it is a maximization problem.
  Let $\{m_n\}$ be the $\NC$ circuit family that computes the measure function.
  Let $\{A_{n, k}\}$ be the circuit family such that
  \begin{itemize}
  \item $R(x, A_{n, k}(x, k)) \leq 1 + \frac{1}{k}$ for each $x$ and $k$,
  \item $\size(A_{n, k}) \leq f(k) n^{O(1)}$,
  \item $\depth(A_{n, k}) \leq f(k) + O(\log^{O(1)} n)$,
  \end{itemize}
  for some circuit-computable function $f$.
  Define the circuit family $\{C_{n, k}\}$ as
  \[
  C_{n, k}(x, k) = 1 \iff m(x, A_{n, k + 1}(x, k + 1)) \leq k,
  \]
  so $C_{n, k}$ outputs 1 if and only if the approximate solution corresponding to parameter $k + 1$ measures less than $k + 1$.
  (The function $m$ is really a circuit as well, chosen from a family of circuits depending on the number of bits in its inputs.)

  The size of $C_{n, k}$ is $O(\size(m) + \size(A_{n, k + 1}))$ and the depth is $O(\depth(m) + \depth(A_{n, k + 1})$.
  For some sufficiently large circuit-computable function $f'$, the size and depth bounds are $f'(k + 1) n^{O(1)}$ and $f'(k + 1) + O(\log^{O(1)} n)$, respectively.
  It remains to show correctness of $C_{n, k}$.

  Let $x$ be a string, let $k$ be a natural number, and let $y = A_{n, k + 1}(x, k + 1)$.
  If $C_{n, k} = 1$, then $m(x, y) \leq k$, so $m^*(k) \leq k$ and therefore $(x, k) \in p \dash O$.
  For the converse, if $C_{n, k} = 0$, then $m(x, y) \geq k + 1$, so
  \[
  m^*(x) \geq \frac{m(x, y)}{1 + \frac{1}{k + 1}} \geq \frac{k + 1}{1 + \frac{1}{k + 1}} = \frac{(k + 1)^2}{k + 2} > k.
  \]
  Thus $(x, k) \notin p \dash O$.
  Therefore, we conclude that $p \dash O$ is in $\para \NC$.
\end{proof}

The converse does not hold: the minimum vertex cover problem is a counterexample because it is in $\para \NC$ \autocite[Theorem~4.5]{bst15}.
\todo{Show an example of an optimization problem whose budget problem is $\P$-complete and whose standard parameterization is in $\para \NC$ but for which no $\ENCAS$ exists.}

Our goal now reduces to finding an optimization problem in $\ENCAS$ whose budget problem is $\P$-complete.
%% There exist problems that are $\P$-complete and have randomized $\FNCAS$ algorithms, so under a derandomization assumption, we can show a problem that fits the requirement.

\begin{definition}[\textsc{Maximum Flow}]
  \mbox{} \\
  \begin{tabular}{r p{9.2cm}}
    \textbf{Instance:} & directed graph $G$, a natural number capacity $c_e$ for each edge $e$, source node $s$, and target node $t$. \\
    \textbf{Solution:} & flow $F$, defined as a real number $F_e$ for each edge $e$ such that $F_e \leq c_e$ and at each vertex the total in-flow is at least the total out-flow. \\
    \textbf{Measure:} & total in-flow at $t$. \\
    \textbf{Type:} & maximization.
  \end{tabular}
\end{definition}

%% edge flows can have weights exponential in n

\begin{theorem}
  If $\NC = \RNC$, then the budget problem for \textsc{Maximum Flow} is $\P$-complete and the standard parameterization is in $\para \NC$.
\end{theorem}
\begin{proof}
  The budget problem for \textsc{Maximum Flow} is $\P$-complete \autocite[Problem~A.4.4]{ghr95}.
  The \textsc{Maximum Flow} problem is in randomized $\FNCAS$ \autocite[Theorem~4.5.2]{dsst97}.
  If $\NC = \RNC$, then randomized $\FNCAS$ equals deterministic $\FNCAS$.
  Thus, the problem is in $\ENCAS$, by \autoref{prop:encas}.
  Finally, the standard parameterization is in $\para \NC$ by \autoref{thm:encasfpp}.
\end{proof}

\todo{Can the randomization part of the RNC algorithm for MaxFlow be absorbed in the fixed-parameter part of the algorithm?}

% The positive linear programming problem is only in NCAS, not FNCAS.
