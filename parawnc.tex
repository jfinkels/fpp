\section{\texorpdfstring{$\para \NC$}{paraNC} is to \texorpdfstring{$\NC$}{NC} as \texorpdfstring{$\para \WNC$}{paraWNC} is to \texorpdfstring{$\NNC$}{NNC}}

\subsection{Definition of \texorpdfstring{$\para \WNC$}{paraWNC}}

\begin{definition}[{$\para \WNC^d$}]
  Let $d$ be a natural number.
  A parameterized problem $(Q, \kappa)$ is in the class $\para \WNC^d$ if there are circuit-computable functions $f$, $g$, and $h$, and a nondeterministic circuit family $\{C_{n, k}\}$ such that for each string $x$,
  \begin{itemize}
  \item $x \in Q$ if and only if $C_{n, k}(x) = 1$, where $n = |x|$ and $k = \kappa(x)$,
  \item $\size(C_{n, k}) \leq f(k) n^{O(1)}$,
  \item $\depth(C_{n, k}) \leq g(k) \log^d n$,
  \item $\nondet(C_{n, k}) \leq h(k) \log n$.
  \end{itemize}
\end{definition}

\subsection{Example problems in \texorpdfstring{$\para \WNC$}{paraWNC}}
\label{sec:rankinwp}

What kind of problems are in the class $\para \WNC$?

\begin{itemize}
\item The parameterized weighted Boolean formula satisfiability problem is in $\para \WNC^1$ \autocite[Theorem~3.6]{est15}.
\item The parameterized feedback vertex set problem is in $\para \WL$ \autocite[Section~4.3]{est15}, which is a subclass of $\para \WNC^2$.
\item The parameterized semigroup rank problem is in $\para \WNL$ \autocite[Theorem~3.12]{est15}, which is a subclass of $\para \WNC^2$.
\end{itemize}

\subsubsection{Small generating set problems}

Let us consider the parameterized group rank problem, a subset of the parameterized semigroup rank problem.

\begin{definition}[$\pgrouprank$]
  \mbox{} \\
  \begin{tabular}{r p{9.2cm}}
    \textbf{Instance:} & finite group $G$ given as a product table, positive integer $k$. \\
    \textbf{Parameter:} & $k$. \\
    \textbf{Question:} & Does $G$ have a generating set of cardinality $k$?
  \end{tabular}
\end{definition}

This problem is highly parallelizable, given that a witness of size $O(k \log n)$ is provided.

\begin{theorem}\label{thm:pgrouprank}
  $\pgrouprank$ is in $\para \WL$.
\end{theorem}
\begin{proof}
  The Turing machine receives $G$ and $k$ as input and a subset $S$ of $k$ group elements as a witness.
  It loops over each element $g$ in $G$ and decides whether $g \in \gen{S}$.
  The machine accepts exactly when all of the subgroup membership tests pass.
  Looping over $n$ elements uses $O(\log n)$ space.
  Deciding whether $g$ is in $\gen{S}$ is the subgroup membership problem, which is in $\SL$, which in turn equals $\L$.
  Thus the overall space usage is $O(\log n)$.
  The size of the witness is $k \log n$, and two-way access is required, since we execute the subgroup membership procedure $n$ times.
  We conclude that this parameterized problem is in $\para \WL$.
\end{proof}

\subsubsection{Weighted satisfiability problems}

Let $\|x\|_1$ denote the Hamming weight (that is, the number of ones) of a binary string.
This problem is the restriction of $\pCSAT$ to bounded depth circuits.

\begin{definition}[$\pNCSAT$]
  \mbox{} \\
  \begin{tabular}{r p{9.2cm}}
    \textbf{Instance:} & Boolean circuit $C$ on $m$ inputs with depth $\log^d m$, natural number $k$. \\
    \textbf{Parameter:} & $k$. \\
    \textbf{Question:} & Is there a binary string $w$ such that $\|w\|_1 = k$ and $C(x) = 1$?
  \end{tabular}
\end{definition}

To prove membership in $\para \WNC^d$, we will use the following \emph{decoder function}.
Since we will also later use its inverse, the \emph{encoder function}, we define it here as well.

\begin{definition}\label{def:coding}
  The \emph{encoder function}, $E_n \colon \{0, 1\}^m \to \{0, 1\}^{\log m}$, is defined by
  \[
  E_m(x) =
  \begin{cases}
    i & \text{if } x \text{ has exactly one 1 at index } i \\
    0^{\log n} & \text{otherwise}.
  \end{cases}
  \]
  for each binary string $x$ of length $m$.
  The \emph{decoder function}, $D_m$, is defined as the inverse of the encoder function.
\end{definition}

\begin{lemma}\label{lem:codingbounds}
  For each natural number $m$, both the encoder and decoder functions are computable by a circuit of size $O(m)$ and depth $O(\log m)$.
\end{lemma}
\begin{proof}
  This can be found in, for example, Lemmas~2.5.3 and 2.5.4 of \autocite{savage98}.
\end{proof}

This theorem is an adaptation of \autocite[Lemma~3.3]{cc97lim}.

\begin{lemma}\label{lem:pkncsatmem}
  For each positive integer $d$, we have $\pNCSAT$ is in $\para \WNC^d$.
  %% Also, the underlying decision problem is in $\NNC^d[\poly]$.
\end{lemma}
\begin{proof}
  %% The underlying decision problem has a straightforward $\NNC^d[\poly]$ algorithm using the depth-universal circuit of \autocite{ch85}, along with a logarithmic depth circuit that enforces the inequality $\|w\|_1 = k$.
  %%The number of nondeterministic bits used by that circuit is $n$, the number of inputs to the circuit $C$, which is bounded above by the size of $C$.
  %%
  %% Membership in $\para \WNC^d$ requires a little more work.
  The algorithm will be composed of several subcircuits.
  \begin{itemize}
  \item Let $U$ be the depth-universal circuit \autocite{ch85} for depth $\log^d m$.
  \item Let $D_{m, k}$ be the function defined by $D_{m, k}(w) = D_m(w_1) \lor \dotsb \lor D_m(w_k)$, where $\lor$ denotes bitwise \textsc{or} for strings of length $m$, the binary string $w_i$ is the $i$th block of $w$ of size $\log m$, and $D_m$ is the decoder function (\autoref{def:coding}).
  %% The circuit $D_{m, k}$ interprets its input as $k$ blocks of length $\log m$ each, then outputs the string containing a one at index $i$ exactly when at least one block is the binary encoding of the natural number $i$.
  \item Let $\Delta$ be the function that takes $k$ blocks of $\log m$ bits each as input and evaluates to true exactly when each pair of blocks of size $\log m$ are distinct. For example, if $m = 4$, then $\Delta(0101) = 0$ but $\Delta(1011) = 1$.
  \end{itemize}
  Define the nonuniform, nondeterministic $\NC^d$ circuit family $\{A_{m, k}\}$ that decides $\pNCSAT$ by
  \[
  A_{n, k}((C, k), w) = U(C, D_{m, k}(w)) \land \Delta(w).
  \]
  In other words, $A_{n, k}$ interprets its witness string $w$ of length $k \log m$ as an encoding of a string of length $m$ containing \emph{exactly} $k$ ones (as enforced by $\Delta$), then evaluates the circuit $C$ on that string.

  This algorithm correctly decides the underlying decision problem.
  For each circuit $C$ of size $n$ and each integer $k$,
  \begin{align*}
    (C, k) \in \pNCSAT & \iff \exists x \in \{0, 1\}^m \colon \|x\|_1 = k \text{ and } C(x) = 1 \\
    & \iff \exists w \in \{0, 1\}^{k \log m} \colon (C(D_{m, k}(w)) \land \Delta(w)) = 1 \\
    & \iff \exists w \in \{0, 1\}^{k \log m} \colon (U(C, (D_{m, k}(w))) \land \Delta(w)) = 1 \\
    & \iff \exists w \in \{0, 1\}^{k \log m} \colon A_{n, k}((C, k), w) = 1.
  \end{align*}
  %% The number of nondeterministic bits used by $A_{n, k}$ is of the form $h(k) \log m$ if we choose $h$ to be the identity function.
  If we assume without loss of generality that $m \leq n$ (by padding with useless gates, for example), then the number of nondeterministic bits used is less than $k \log m$, which is of the form $h(k) \log n$.

  For the size and depth bounds, we need to determine the size and depth of the circuits for $U$, $D_{m, k}$, and $\Delta$.
  \begin{itemize}
  \item The depth-universal circuit $U$ has size $m^{O(1)}$ and depth $O(\log^d n)$, which is $O(\log^d m)$ because we can assume without loss of generality that the size of $C$ is at least the number of its inputs.
  \item
    Decoding a single block of size $\log m$ requires size $O(m)$ and depth $O(\log m)$ by \autoref{lem:codingbounds}.
    Decoding all $k$ blocks of size $\log m$ requires $k$ copies of that subcircuit, along with an \textsc{or} tree for each of the $m$ output bits.
    Thus the size of $D_{m, k}$ is $O(k m + m \log k)$ and the depth $O(\log m + \log k)$.
  \item
    Comparing two binary strings of length $\log m$ for inequality requires $O(\log m)$ size and $O(\log \log m)$ depth.
    Comparing all $\binom{k}{2}$ pairs of blocks and requiring they are all distinct thus requires a circuit of size $O(\binom{k}{2} \log m)$ and depth $O(\log m + \log k)$.
  \end{itemize}
  Hence the overall size of $A_{n, k}$ is of the form $f(k) m^{O(1)}$ and the depth $f(k) + \log^d m$.
  Since we can assume without loss of generality that $m \leq n$, we get size and depth bounds with the appropriate dependence on $k$ and $n$.

  Since we have shown a $\para \WNC^d$ circuit family deciding $\pNCSAT$, we conclude that the problem is in $\para \WNC^d$.
\end{proof}

\todo{Show a miniaturization of this problem?}

This technique also provides some alternate proofs that parameterized weighted Boolean formula satisfiability problem, denoted $\pFSAT$, is in $\para \WNC^1$, implicit in \autocite[Theorem~3.6]{est15}.
%% \todo{
%%   If we can show a reduction from parameterized formula satisfability to parameterized $\NC^1$ satisfiability, then we have yet another proof, although that's really the same as the one below.
%%   Each formula has an equivalent log depth circuit, but I need to ensure it is computable in $\AC^0$ (or $\FO$).
%% }
The first proof is a reduction to $p \dash k \dash \NC^1 \textsc{Csat}$, and requires two auxiliary algorithms.

\begin{lemma}[{\autocite{lsh65}}]
  There is an $\NC^1$ algorithm that, given a binary tree $T$ on $n$ leaves, finds a subtree on $s$ leaves, where $\frac{1}{3} n \leq s \leq \frac{2}{3} n$.
\end{lemma}
\begin{proof}
  The naïve implementation of the procedure from \autocite{lsh65} would require repeatedly computing the number of leaves in each subtree.
  Instead, we compute the number of leaves in each subtree once at the beginning of the algorithm.

  The algorithm proceeds as follows.
  Compute $\frac{1}{3} n$ and $\frac{2}{3} n$.
  Compute for each node, the number of leaves in the subtree rooted at that node.
  Output the subtree that is within the bounds (one must exist by a counting argument).

  Division and multiplication of integers of length $\log n$ can certainly be done by a circuit of size $O(n)$ and depth $O(\log n)$.
  Computing the number of leaves in each subtree can be done by first computing the number of direct children of each node with a circuit of size $O(n)$ and depth $O(1)$, then performing a prefix-sum operation (only on those prefixes that are paths in the tree) with size $O(n)$ and depth $O(\log n)$ (see \autocite[Theorem~2.6.1]{savage98} for a prefix-sum implementation).
  Comparing each of these numbers to the computed upper and lower bounds requires size $O(n \log n)$ and depth $O(\log \log n)$.
  Finally, outputing the node requires depth $O(1)$, or, if the entire subtree is desired, depth $O(\log n)$ by using a path-doubling algorithm to find each node in the subtree rooted at the given node.

  Overall, the algorithm can be implemented by a circuit of size $n^{O(1)}$ and depth $O(\log n)$.
\end{proof}

\todo{find a copy of \autocite{spira71}}

\begin{lemma}[{\autocite{spira71} (see also \autocite[Theorem~2]{bb94})}]\label{lem:logformula}
  There is an $\NC^1$ algorithm that transforms a Boolean formula of size $n$ into a Boolean formula of size $n^{O(1)}$ and depth $O(\log n)$.
\end{lemma}
\begin{proof}
  Let $A'$ be the algorithm of the previous lemma that finds a subtree of size approximately $n / 2$.
  Construct the algorithm $A$ that proceeds as follows on input $\phi$ of size $n$.
  \todo{Modify this, because it doesn't work: computing A' repeatedly is too much; we can just compute the mapping from node to size of subtree and update it when replacing $\psi$ with 1 or 0.}
  \begin{itemize}
  \item Let $\psi = A'(\phi)$.
  \item Let $\phi' = (\psi \land \phi|_{\psi = 1}) \lor (\lnot \psi \land \phi|_{\psi = 0})$, where $\phi|_{\psi = b}$ denotes the formula $\phi$ with each instance of $\psi$ replaced with the constant $b$.
  \item Apply $A$ recursively to each of the four subformulas in $\phi'$ and return the resulting formula.
  \end{itemize}
  \todo{fill me in}
\end{proof}

\begin{lemma}[{\autocite[Theorem~3.6]{est15}}]\label{lem:pfsatmem}
  $\pFSAT$ is in $\para \WNC^1$.
  %% Also, the underlying decision problem is in $\NNC^1[\poly]$.
\end{lemma}
\begin{proof}
  We show a $\para \NC^1$ many-one reduction to the weighted circuit satisfiability for logarithmic depth circuits.
  By \autoref{lem:logformula}, each Boolean formula $\phi$ of size $n$ can be transformed into an equivalent Boolean formula $\phi'$ of depth $O(\log n)$.
  From there, we simply output the circuit representation $C$ of that logarithmic depth formula.
  Furthermore, the formula $\phi'$ has the property that $\phi(x) = \phi'(x)$ for each assignment $x$, and the circuit $C$ has the property that $\phi'(x) = C(x)$ for each $x$.
  Thus the reduction is correct: $\phi$ has a satisfying assignment of weight exactly $k$ if and only if $\phi'$ does if and only if $C$ does.
  Since the reduction is computable in $\NC^1$ and the parameter is identical, it is trivially computable in $\para \NC^1$.
\end{proof}

The second proof directly implements the decoder function of \autoref{def:coding} as a Boolean formula.

\begin{proof}[Alternate proof of \autoref{lem:pfsatmem}]
  The underlying decision problem has a straightforward $\NNC^1[\poly]$ algorithm using the $\NC^1$ algorithm for Boolean formula satisfiability \autocites{buss87, bcgr92}, along with a logarithmic depth circuit that enforces the inequality $\|w\|_1 = k$.
  The number of nondeterministic bits used by the circuit is the number of variables in the formula, which is bounded above by a polynomial in the size of the formula.

  We adapt the ``$k \log n$ trick'' from circuit inputs to formula variables.
  This requires a reimplementation of the decoder function of \autoref{def:coding} as a function on Boolean variables.

  Similar to the proof of \autoref{lem:pkncsatmem}, the algorithm involves composing a decoder and an algorithm for evaluating a Boolean formula, after nondeterministically choosing a witness.
  Again, we use some subcircuits:
  \begin{itemize}
  \item Let $U$ be the $\NC^1$ algorithm for evaluating a Boolean formula \autocites{buss87, bcgr92}.
  \item
    Let $M_i$ be the function on $\log m$ that outputs the $i$th minterm of its input variables (for example, $M_4(w_1, w_2, w_3) = w_1 \land \lnot w_2 \land \lnot w_3$).
    This function acts like the decoder in \autoref{lem:pkncsatmem}.
  \item
    Let $R_{m, k}(\phi)$ be the function that replaces each instance of a variable $x_i$ in a Boolean formula $\phi$ on $m$ variables with $\vee_{j = 1}^k M_i(\vec{v}_j)$, where $\vec{v}_j$ denotes the $j$th block of size $\log m$ in a tuple of $k \log m$ new variables $v_1, \dotsc, v_{k \log m}$.
  \item Let $\Delta$ be the function that takes $k$ blocks of $\log m$ bits each as input and evaluates to true exactly when each pair of blocks of size $\log m$ are distinct.
  \end{itemize}
  Define the nonuniform, nondeterministic $\NC^1$ circuit family $\{A_{n, k}\}$ that decides $\pFSAT$ by
  \[
  A_{n, k}((\phi, k), w) = U(R_{m, k}(\phi), w) \land \Delta(w).
  \]
  In other words, $A_{n, k}$ interprets its witness string $w$ of length $k \log m$ as the \emph{encoding of an assignment} to the variables of $\phi$ in which exactly $k$ variables are set to true, then evaluates the formula $\phi$ with respect to the decoded assignment.

  This algorithm correctly decides the weighted Boolean formula satisfiability problem.
  Similar to the proof of \autoref{lem:pkncsatmem}, $\phi$ has a satisfying assignment of weight exactly $k$ if and only if $\phi'$ has a satisfying assignment (of arbitrary weight).
  If we assume without loss of generality that $m \leq n$ (by padding with tautological conjuncts, for example), then the number of nondeterministic bits used is less than $k \log m$, which is of the form $h(k) \log n$.

  The algorithm also has the appropriate size and depth bounds.
  Since each minterm is of size exactly $\log m$ and each variable $x_i$ is represented the disjunction of $k$ such minterms, the new formula $\phi'$ is of size $|\phi| k \log m$, which is just $n k \log m$.
  \begin{itemize}
  \item
    Each variable $x_i$ can be replaced in parallel, and within that replacement, each disjunct $M_i(\vec{w}_j)$ can be replaced in parallel as well.
    The circuit that writes $M_i(\vec{w}_j)$ is a circuit of size $O(\log m)$ and depth $O(\log m)$ (for using the index $i$ as a selector in a multiplexer), so $R_{m, k}$ can be implemented by a circuit of size $O(n k \log m)$ and depth $O(\log m)$.
  \item
    The circuit $U$ receives a formula of size $O(n k \log m)$ and an assignment of length $O(k \log m)$.
    Since $U$ is a circuit of size polynomial in its the size of its input, its size is $O((n k \log m)^c)$ for some constant $c$.
    Similarly, its depth is $O(\log^d (n k \log m))$.
  \item As in \autoref{lem:pkncsatmem}, the circuit for $\Delta$ is of size $O(\binom{k}{2} \log m)$ and depth $O(\log m + \log k)$.
  \end{itemize}
  The overall size and depth of the circuit $A_{n, k}$ are therefore of the form $f(k) n^{O(1)}$ and $f(k) + O(\log^d n)$, respectively.
  \todo{Add info in the introduction about why we can assume, since $k$ is smaller than $O(\log n)$, that $O(\log^d nk) \leq f(k) + O(\log^d n)$.}

  At this point, we have shown a correct $\para \WNC^1$ algorithm for $\pFSAT$.
\end{proof}

The same proof works for the problems of deciding whether a circuit or a formula has a satisfying assignment of weight \emph{at most} $k$, as well (one could even remove the $\Delta$ subcircuit entirely, but that is not necessary).

\subsection{Completeness in \texorpdfstring{$\para \WNC$}{paraWNC}}

We saw that the parameterized weighted Boolean formula satisfiability problem, $\pFSAT$, is in $\para \WNC^1$ in \autoref{lem:pfsatmem}.
In fact, it is complete for $\para \WNC^1$ under $\para \FO$ many-one reductions \autocite[Theorem~3.6]{est15}.
\todo{Can we use the ``template on top of a verification language'' elsewhere?}
The parameterized weighted Boolean circuit satisfiability problem, denoted $\pCSAT$, is the same problem with Boolean circuits instead of Boolean formulas; this problem is complete for $\para \WP$ under $\para \FO$ many-one reductions by a similar proof.
It makes sense to expect, then, that the problems $\pNCSAT$ may be complete for the classes $\para \WNC^d$, for each positive integer $d$.

%% One strategy for proving complete problems for the $\para \WNC^d$ classes might be to adapt the proof of the $\para \WNC^1$-completeness of $\pFSAT$.
%% Since Boolean formulas are equivalent to Boolean circuits of logarithmic depth \todo{add citation}, the $\pFSAT$ can be viewed as $p \dash k \dash \NC^1 \textsc{CSat}$.
%% However, the proof of \autocite[Theorem~3.6]{est15} does not translate directly to the setting of bounded depth circuit satisfiability (their reduction would yield a circuit of depth greater than polylogarithmic in the number of inputs).
In order to prove completeness for these problems, we use the strategy from \autocite[Theorem~3.6]{cc97lim}, which relies on the ``$k \log n$ trick'' (see \autocite[Corollary~3.13]{fg06}, or the origin \autocite{adf95}).
There the authors prove that the decision problem underlying $\pNCSAT$ is complete for $\NNC^d[k \log n]$.
As we will see in the next section (and repeatedly throughout this paper), parameterized complexity and limited nondeterminism in decision complexity are closely related.

\begin{theorem}\label{thm:parawncdcomplete}
  For each positive integer $d$, we have $\pNCSAT$ is complete for $\para \WNC^d$ under $\para \NC^1$ many-one reductions.
  %% Also, the underlying decision problem is in $\NNC^d[\poly]$.
\end{theorem}
\begin{proof}
  Membership in $\para \WNC^d$ was proven in \autoref{lem:pkncsatmem}.

  Suppose $(Q, \kappa)$ is in $\para \WNC^d$, so there is a nonuniform $\NC^d$ circuit family $\{C_{n, k}\}$ and circuit-computable functions $f$ and $h$ such that
  \begin{itemize}
  \item $x \in Q$ if and only if $C_{n, k}(x) = 1$,
  \item $\size(C_{n, k}) \leq f(k) n^{O(1)}$,
  \item $\depth(C_{n, k}) \leq f(k) + O(\log^d n)$,
  \item $\nondet(C_{n, k}) \leq h(k) \log n$.
  \end{itemize}
  On input $x$ of length $n$, let $C_x$ denote the circuit $C_{n, k}$ with $x$ hardcoded as its first $n$ inputs.
  Thus $C_x$ is a circuit with $h(k) \log n$ inputs such that $x \in Q$ if and only if $C_x$ is satisfiable.
  Let $E_{n, k}$ denote the function defined by $E_{n, k}(w) = E_n(w_1) \circ \dotsb \circ E_n(w_{h(k)})$, where $\circ$ denotes string concatenation, $E_n$ is the encoder function of \autoref{def:coding}, and $w_i$ is the $i$th block of size $n$ in the string $w$, for each string $w$ of length $h(k) n$.
  The reduction is then $x \mapsto (C', h(k))$, where $C' = C_x \circ E_{n, k}$.

  The circuit $C'$ is of the correct form to be an input to $\pNCSAT$.
  The size of $C'$ is
  \begin{align*}
    \size(C') & = \size(E_{n, k}) + \size(C_x) \\
    & \leq h(k) \size(E_n) + f(k) n^{O(1)} \\
    & \leq h(k) O(n) + f(k) n^{O(1)}
  \end{align*}
  and the depth
  \begin{align*}
    \depth(C') & = \depth(E_{n, k}) + \depth(C_x) \\
    & \leq \depth(E_n) + (f(k) + \log^d n) \\
    & \leq O(\log n) + f(k) + \log^d n \\
    & \leq f(k) + O(\log^d n).
  \end{align*}
  The number of inputs to $C'$ is $h(k) n$, so we need the size to be polynomial in $h(k) n$ and the depth to be $\log^d (h(k) n)$.
  The size bound is satisfied if we choose $h(k)$ so that $f(k) \leq h(k)$ and the depth bound satisfied if we choose $h(k)$ so that $f(k) \leq \log^d h(k)$, ignoring some constants that can be incorporated into the function $h$.
  We can choose $h$ this way without loss of generality, because choosing a larger $h$ does not affect membership of $(Q, \kappa)$ in $\para \WNC^d$.
  (This does, however, cause the nondeterminism upper bound of the problem $(Q, \kappa)$ to be extremely loose and the dependence on $k$ in the size and depth bounds of $C'$ to be extremely high, but it is technically sufficient.)

  The reduction is a correct many-one reduction between the underlying decision problems.
  Suppose $x \in Q$, so $C_x$ is satisfiable.
  Since $E_n$ is surjective, so is $E_{n, k}$, hence there is a string $w$ (of length $h(k) n$) such that $C'(w) = 1$.
  Furthermore, this string satisfies $\|w\|_1 = h(k)$ since all preimages of $E_{n, k}$ satisfy this equality.%
  \footnote{%
    Technically, we need to guarantee the satisfying input to $C_x$ has no all-zero blocks to make this statement.
    For each parameterized problem $(Q, \kappa)$ there is another equivalent problem that satisfies such a requirement on the witnesses for $Q$ with no change in complexity.%
  }
  Therefore, $C'$ has a satisfying assignment of Hamming weight exactly $h(k)$.
  For the converse, suppose $C'$ has a satisfying assignment $w$ of weight exactly $h(k)$.
  Then there is a satisfying assignment of length $h(k) \log n$ for $C_x$, namely $E_{n, k}(w)$.
  Therefore, $x$ is in $Q$.

  The reduction is $\para \NC^1$-computable.
  The size of the circuit computing the reduction is simply the size of the output, which is $\size(C') + \size(h_k)$, where $h_k$ denotes the circuit computing $h$ on inputs of size $k$.
  Both addends are of the form $f'(k) n^{O(1)}$ for some circuit-computable function $f'$.
  The depth of the circuit computing the reduction is dominated by the depth of the $h_k$, which is bounded above by $f'(k) + \log n$ for some function $f'$.
  Thus the size and depth requirements for the reduction are met.
  Finally, if $R$ denotes the reduction and $\kappa'$ denotes the parameterization for $\pNCSAT$,
  \[
  \kappa'(R(x)) \leq \kappa'((C', h(\kappa(x)))) = h(\kappa(x)).
  \]
  Since $h$ is circuit-computable by hypothesis, the reduction meets the parameterization bound.

  Since we have shown a correct $\para \NC^1$ many-one reduction from an arbitrary parameterized problem in $\para \WNC^d$ to $\pNCSAT$, we conclude that $\pNCSAT$ is complete for the class.
\end{proof}

This technique also gives us an alternate proof of the completeness $\pFSAT$ \autocite[Theorem~3.6]{est15}.
However, we can prove only completeness under the $\para \NC^1$ reduction instead of the tighter $\para \FO$ reduction.

\begin{theorem}\label{thm:pfsatcomplete}
  $\pFSAT$ is complete for $\para \WNC^1$ under $\para \NC^1$ many-one reductions.
  %% Also, the underlying decision problem is in $\NNC^1[\poly]$.
\end{theorem}
\begin{proof}
  Membership in $\para \WNC^1$ was proven in \autoref{lem:pfsatmem}.

  To prove hardness, it suffices to show a $\para \NC^1$ reduction from $p \dash k \dash \NC^1 \textsc{CSat}$ to $\pFSAT$, since the circuit problem is complete for $\para \WNC^1$ by the previous theorem.
  The reduction is $(C, k) \mapsto (\phi, k)$, where $\phi$ is the standard reduction for logarithmic depth circuits to polynomial size Boolean formulas.
  This reduction recursively replaces each subcircuit with its corresponding formula, using the input gates of the circuit as the variables for the formula.
  Since the depth of the (fan-in 2) circuit is $O(\log n)$, where $n$ is the number if inputs to $C$, the size of the formula is $2^{O(\log n)}$, which is polynomial in $n$.
  Furthermore, since the variables appearing in the formula are exactly the input gates of the circuit, the circuit has a satisfying assignment of weight $k$ if and only if the formula has a satisfying assignment of weight $k$.
  Finally, the reduction is computable by circuit of size $O(|\phi|)$ and depth $O(\depth(C))$, so it is computable in $\NC^1$ \todo{someone verify this}.
  Since furthermore the parameter is unchanged by the reduction, the $\NC^1$-computable function trivially induces a $\para \NC^1$ reduction.
\end{proof}

Now we have a full interpolation for the inclusion chain
\[
\para \WNC^1 \subseteq \para \WNC^2 \subseteq \dotsb \subseteq \para \WP,
\]
via the chain of parameterized reductions between corresponding complete problems
\[
\pFSAT \leq p \dash k \dash \NC^2 \textsc{CSat} \leq \dotsb \leq p \dash \textsc{CSat}.
\]
Whether the $\pNCSAT$ problems are complete under $\para \FO$ many-one reductions remains open.

\subsection{Does \texorpdfstring{$\para \NC$}{paraNC} equal \texorpdfstring{$\para \WNC$}{paraWNC}?}
\label{sec:fppwpp}

The main theorem in this section, \autoref{thm:ncwnc}, provides evidence that the parameterized complexity classes $\para \NC$ and $\para \WNC$ are distinct.
Compare it with the following similar theorems (here $\beta$ indicates one-way access to the witness as opposed to $\W$, which indicates two-way access).
\begin{itemize}
\item $\para \P = \para \WP$ if and only if $\P = \NP[\omega(\log n)]$ \autocite[Theorem~3.29]{fg06}.
\item $\para \L = \para \beta \L$ if and only if $\L = \NL[\omega(\log n)]$ \autocite[Theorem~15]{cm14}.
\end{itemize}
\todo{Can we generalize all of these theorems?}

This lemma is an adaptation of \autocite[Lemma~3.24]{fg06}.
It is the ``opposite'' of \autoref{lem:upperinverse}.

\begin{lemma}\label{lem:lowerinverse}
  Suppose $d$ is a positive integer.
  For each increasing, circuit-computable function $f$, there is a function $i_{f, d}$ such that
  \[
  f(i_{f, d}(n)) \leq \min(n, \log^d n)
  \]
  for each $n \geq f(1)$.
  Furthermore, $i_{f, d}$ is circuit-computable, nondecreasing, unbounded.
\end{lemma}
\begin{proof}
  Define $i_{f, d}$ by
  \[
  i_{f, d}(n) = \max\{ j \in \mathbb{N} \, | \, f(j) \leq \min(n, \log^d n) \}.
  \]
  For the boundary case where $n$ is smaller than $f(1)$, define $i_{f, d}(n) = 1$.
  It is straightforward to prove that this function is nondecreasing and unbounded.
  (The integer $d$ must be greater than zero to guarantee $i_{f, d}$ is unbounded.)

  To compute $i_{f, d}$, we use the fact that $f$ is increasing to perform a binary search on the values $f(1), \dotsc, f(\min(n, \log^d n))$ and to determine the largest $j$ such that $f(j) \leq \min(n, \log^d n)$.
  In the worst case, there will be at most $\log n$ comparison subcircuits, each requiring a computation of $f$, so the overall depth of the circuit computing $i_{f, d}$ is $O(\depth(f) \log n)$ and the size is $O(\size(f) \log n)$.
\end{proof}

This theorem is an adaptation of \autocite[Theorem~3.29]{fg06}.
For any computable function $i$, let $\iNCSAT$ denote the problem of deciding whether a depth $O(\log^d n)$ circuit is satisfiable by an input of Hamming weight $k$, where $k$ is at most $i(n)$.
This problem is complete for $\NNC^d[i(n) \log n]$ under logarithmic space many-one reductions \autocite[Theorem~3.6]{cc97lim} (\todo{they technically use the problem ``has a satisfying assignment of weight $\leq i(n)$'', but I believe this is equivalent.}).

\begin{theorem}\label{thm:ncwnc}
  Suppose $d$ is a positive integer.
  $\para \NC^d = \para \WNC^d$ if and only if there is a circuit-computable, nondecreasing, unbounded function $i$ such that $\NC^d = \NNC^d[i(n) \log n]$.
\end{theorem}
\begin{proof}
  First we prove the reverse implication.
  Assume $\NC^d = \NNC^d[i(n) \log n]$, for some $i$.
  Since $\pNCSAT$ is complete for $\para \WNC^d$ by \autoref{thm:parawncdcomplete}, it suffices to show that this problem is in $\para \NC^d$, which necessitates the claimed collapse.
  By the comments preceding this theorem, the problem $\iNCSAT$ is in $\NNC^d[i(n) \log n]$, which means it is also in $\NC^d$ by assumption.
  If $(Q, \kappa)$ denotes $\pNCSAT$ and $i(n) \dash Q$ denotes $\iNCSAT$, then the former is in $\para \NC^d$ by \autoref{cor:sprself}.

  Now we prove the forward implication.
  Assume $\para \NC^d = \para \WNC^d$.
  By \autoref{thm:parawncdcomplete}, the parameterized problem $\pNCSAT$ is now in $\para \NC^d$.
  Let $\{A_{n, k}\}$ be the $\para \NC^d$ circuit family that decides this problem, with size bound $f(k) n^{O(1)}$ and depth bound $f(k) + \log^d n$, for some circuit-computable function $f$.
  Assume without loss of generality that $f$ is increasing.
  Choose $i$ to be the ``lower inverse'' function $i_{f, d}$ guaranteed by \autoref{lem:lowerinverse}.

  The decision problem $i(n) \dash \NC^d \dash \textsc{CSat}$ is complete for $\NNC^d[i(n) \log n]$ under logarithmic space many-one reductions by \autocite[Theorem~3.6]{cc97lim}, so it suffices to show that this problem is in $\NC^d$.
  \todo{Need to show the case $d = 1$ works as well.}
  The circuit family $\{A'_n\}$ defined by $A'_n(C) = A_{n, i(n)}((C, i(n)))$ witnesses this membership:
  \[
  \size(A'_n) = \size(A_{n, i(n)}) \leq f(i(n)) n^{O(1)} \leq n^{O(1)}
  \]
  and
  \[
  \depth(A_n) = \depth(A_{n, i(n)}) \leq f(i(n)) + O(\log^d n) \leq O(\log^d n),
  \]
  by the way $i$ was chosen.
  The correctness of this algorithm follows from the correctness of $A_{n, k}$.
  Since we have shown an $\NC^d$ algorithm for an $\NNC^d[i(n) \log n]$-complete problem, we conclude that $\NC^d = \NNC^d[i(n) \log n]$.
\end{proof}

As a corollary, any problem complete for $\para \WNC^d$ is not in $\para \NC^d$ unless $\NC^d = \NNC^d[i(n) \log n]$.

\begin{corollary}
  For each positive integer $d$, we have $\pNCSAT$ is in $\para \NC^d$ if and only if there is a circuit-computable, nondecreasing, unbounded function $i$ such that $\NC^d = \NNC^d[i(n) \log n]$.
  (Here we equate $\NC^1 \dash \textsc{CSat}$ with $\textsc{FSat}$.)
\end{corollary}

\subsection{Is \texorpdfstring{$\para \WNC$}{paraWNC} in \texorpdfstring{$\para \P$}{paraP}?}
\label{sec:wncp}

The previous section demonstrates that it is unlikely that $\para \WNC$ is contained in $\para \NC$.
Does the weaker inclusion of $\para \WNC$ in $\para \P$ also have negative consequences, albeit less severe ones?
This conjecture relaxes \autoref{thm:ncwnc}.

\begin{conjecture}\label{con:wncp}
  $\para \WNC^d \subseteq \para \P$ if and only if $\NNC^d[i(n) \log n] \subseteq \P$.
\end{conjecture}

Here is an alternate approach that connects a collapse in parameterized complexity classes to one in classical complexity classes.

\begin{theorem}[{\autocite[Corollary~3.7]{est15}}]
  If $\NC^1 = \P$, then $\W[\SAT] = \para \WP$.
\end{theorem}

As in \autoref{thm:parawncdcomplete}, we wish to generalize this theorem so that we have an interpolation between $\NC^1$ and $\P$.
%% Writing ``$\P$'', the upper limit of the interpolation, in place of ``$\NC^d$'' on the left gives $\para \WP = \para \WP$ on the right, a tautology.

\begin{conjecture}
  For each positive integer $d$, if $\NC^d = \P$, then $\W[\NC^d \SAT] = \para \WP$.
\end{conjecture}

An alternative approach is to demonstrate related collapses in parameterized complexity classes.

\begin{theorem}[{\autocite[Corollary~3.8]{est15}}]
  $\para \WNC^1 \subseteq \para \P$ if and only if $\para \P = \W[\SAT]$.
\end{theorem}

Here we interpolate between $\para \WNC^1$ and $\para \WP$.
%% Writing ``$\para \WP$'' in place of ``$\para \WNC^d$'' on the left, the upper limit of the interpolation, gives $\para \P = \para \WP$ on the right, a tautology.

\begin{conjecture}
  For each positive integer $d$, we have $\para \WNC^d \subseteq \para \P$ if and only if $\para \P = \W[\NC^d \SAT]$.
\end{conjecture}

\subsection{Does \texorpdfstring{$\para \WNC$}{paraWNC} equal \texorpdfstring{$\para \WP$}{paraWP}?}
\label{sec:wncwp}

\begin{conjecture}
  For each positive integer $d$, if $\para \WNC^d = \para \WP$, then $\NC^d = \P$.
\end{conjecture}
\begin{justification}
  $p \dash k \dash \textsc{CSat}$ is $\para \WP$-complete and $p \dash \textsc{FSat}$ is $\para \WNC^1$-complete \autocite[Theorem~3.6]{est15}.
  A reduction from the former to the latter would be very surprising.
\end{justification}
